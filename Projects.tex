% File name     : Projects.tex
% Description   : It places the "Projects" CVSection in this script
% Author        : Frederique Hsu
% Date          : Sun.  13 Oct. 2024
%
%


\documentclass[./CV]{subfiles}


\begin{document}
    
    \engcvsection{\faPencilSquareO{} Projects}
    
    \projectsubsection{PoissonSoft Constraint Solver Engine}{PoissonSoft}
    
    \projectdetail{
        \begin{tabular*}{1\linewidth}{@{}p{0.7\linewidth} @{}p{0.3\linewidth}}
            \fbox{Proj. description and Responsibility: }
            \vspace{5pt}
            
            \faCaretRight\quad{}Developed the domestic CAD constraint solver engine, compared with the benchmark Siemens D-Cubed.
            
            \faCaretRight\quad{}Design and implement some local APIs as below and cloud interfaces {\texttt{PSCSCloudAPI}} in C++17,
            and the 
            
            replaying feature of solving log scripts.
            
            \texttt{PSCSApiSolver2DEngine::isLicenseValid}, \texttt{PSCSApiSolver2D::measureDimension}, \texttt{incrementalEvaluate},
            
            \texttt{dragging}, \texttt{overConstraintAnalyze} etc.
            
            \faCaretRight\quad{}Coding: 55\%, Testing and performance parsing: 15\%, Write Wiki document and design solution: 20\%,
            
            Fixing issues: 10\%
            
            \faCaretRight\quad{}Develop the visual utility Sketcher and PSCAD.3DViewer (with Qt widgets, Qt3DRender) and framework.cad
            
            \faCaretRight\quad{}Studying on contrast the innovative solving algorithms of open-source OpenCascade, compare the difference
            
            between OCCT, PSCS Engine and D-Cubed in terms of implementation and solving.
            
            \faCaretRight\quad{}Study how to vectorize the large-scale Jacobian matrix, combining with OpenCL heterogeneous parallel 
            
            computing, deploy it onto the GPU to improve parallelism.
            
            \faCaretRight\quad{}Maintain the CMake scripts, automate the Build/Test/Pack/Deploy workflow, have PSCS engine worked 
            
            on Windows/Linux/macOS cross-platform and HuaweiCloud, meanwhile compatible with x86\_64 CPU and 
            
            Huawei Kunpeng ARM CPU.
            
            & 
            
            \fbox{Tech. stack I used: }
            \vspace{5pt}
            
            Graph model connected component\quad{}
            
            nlohmann-json serialization \& deserialization\quad{}
            
            Eigen MatrixLib\qquad{} Qt3DRender\quad{}
            
            OpenCascade solving\quad{}
            
            OpenCL heterogeneous parallel computing\quad{}
            
            GoogleTest\qquad{} Vistor Pattern\quad{}
            
            CMake\quad{}
            
            \vspace{20pt}
            
            \fbox{Some important algorithms: }
            \vspace{5pt}
            
            Newton-Raphson iteration method\quad{}
            
            Trust-Region method\quad{}\\
        \end{tabular*}
    }

    \vspace{-5pt}
    \textcolor{blue!30}{\hrule}
    \projectsubsection{Autonomous Driving Perception/Road Model module}{BMW}
    
    \projectdetail{
        \begin{tabular*}{1\linewidth}{@{}p{0.7\linewidth} @{}p{0.3\linewidth}}
            \fbox{Proj. description and Responsibility: }
            \vspace{5pt}
            
            \faCaretRight\quad{}Develop the Road Model module of Perception system of L2 Autonomous Driving for BMW 3series, 
            
            to meet the requirement of China market.
            
            \faCaretRight\quad{}Parse in-depth the trajectories of surrounding vehicles, which were sensored by Lidar, mmRadar and camera
            
            array. Utilizing Kalmann Filter algo. to fit/calibrate/identify those trajectories, then aggregate them into 
            
            Left/Ego/Right lane bunches, finally fusion out the virtual lanes, relative to the physical lanes.
            
            \faCaretRight\quad{}Playback the fitted trajectories and the real-time video stream on ROS RViz emulator, check whether the
            
            fitted trajectories are approximating the genius vehicle's trajectories, especially on the curved road or 
            
            lane-changing, in some extreme scenarios of road forks and merging, to filter out the distorted trajectories.
            
            \faCaretRight\quad{}Coding: 40\%, Testing(Unit test + Integrated test + HIL + SIL): 30\%, Design the solution: 15\%, 
            
            Experiment on testing ground and highway: 15\%
            
            \faCaretRight\quad{}Generate the random driving scenarios in Python script, then carry out the Fuzzy Test, log the test failures.
            
            \faCaretRight\quad{}Gather the video stream which are collected from experiment cars and BMW vehicles, stored in the data center,
            
            orchestrate the Docker containers to testing pipeline, make the extensive testing for each feature and full scenarios.
            
            \faCaretRight\quad{}Write the publisher/subscriber topics in the ROS environment, develop the robot nodes for EgoVehicle and
            
            surrounding vehicles, Simulate in the ROS RViz and visualize the real effect.
            
            & 
            
            \fbox{Tech. stack I used: }
            \vspace{5pt}
            
            Google Bazel incremental build system\quad{}
            
            GoogleTest\quad{}
            
            ROS RViz emulator\quad{}
            
            Docker containers orchestrating\quad{}
            
            \vspace{20pt}
            
            \fbox{Some important algorithms: }
            \vspace{5pt}
            
            Kalmann Filter algorithm\quad{}
            
            Kruskal, Prim algorithm to find the min splay tree, aggregate the trajectory bunches.
        \end{tabular*}
    }
\end{document}